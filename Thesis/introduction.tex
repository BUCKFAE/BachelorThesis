%!TEX root = thesis.tex

\chapter{Introduction}
\label{ch:introduction}

\begin{wrapfigure}{r}{0.5\textwidth}
    \begin{center}
      \includegraphics[width=0.48\textwidth]{images/pikachu.png}
    \end{center}
    \caption{Pikachu, one of the most popular Pokémon~\autocite{Fandom:AshsPikachu}}
    \label{fig:pikachu-image}
\end{wrapfigure}
Pokémon (an abbreviation for \textbf{Pocket Monsters}) is a media franchise managed my \textit{The Pokémon Company}, a company
founded by \textit{Nintendo}, \textit{Game Freak} and \textit{Creatures}~\autocite{Wikipedia:Pokemon}. Figure \ref{fig:pikachu-image} shows an image of
\textit{Pikachu}, one of the most popular Pokémon. Among multiple movies and series, a large variety of Pokémon games was released, starting with
\textit{Pokémon Red} and \textit{Pokémon Blue} which were released on the 28. September 1998, also known as \ac{GEN1}-games. In order to promote
trading between players, \textit{Nintendo} publishes two very similar games at the same time. 
Both releases differ in a few minor story differences and in the available Pokémon. Therefore, in order to collect all available Pokémon,
players either have to buy two copies of the game or trade with a player that owns the counterpart. As of writing, there are eight major
releases, also called \textit{mainline games} with the latest being \textit{Pokémon Sword} and \textit{Pokémon Shield}. While the graphics
evolved a lot over the years (see figures \ref{fig:red0}, \ref{fig:red1}, \ref{fig:sword0} and \ref{fig:sword1}), the key concept of the 
game remained mostly unchanged.
\begin{figure}[ht]
  \centering
  \begin{minipage}{.5\textwidth}
    \centering
    \includegraphics[width=.95\linewidth]{images/Red-0.jpg}
    \captionof{figure}{Exploring the map in \\Pokémon Red}
    \label{fig:red0}
  \end{minipage}%
  \begin{minipage}{.5\textwidth}
    \centering
    \includegraphics[width=.95\linewidth]{images/Red-1.jpg}
    \captionof{figure}{Fighting another trainer in \\Pokémon Red}
    \label{fig:red1}
  \end{minipage}
  \caption*{Image source: \href{https://www.nintendo.de/Spiele/Game-Boy/Pokemon-Rote-Edition-266109.html}{nintendo.de}}
\end{figure}
The player starts his journey in his hometown to become \textit{Pokémon Champion} which is the highest known 
level of rank for a Pokémon trainer \todo{https://bulbapedia.bulbagarden.net/wiki/Pokémon_Champion}. In order
to achieve this goal, the player has to create a team of up to six individual Pokémon which he then trains 
to unleash their full potential. 
\begin{figure}
  \centering
  \begin{minipage}{.5\textwidth}
    \centering
    \includegraphics[width=.95\linewidth]{images/Sword-0.jpg}
    \captionof{figure}{Exploring the map in \\Pokémon Sword. \\ 
      Image source: \href{https://swordshield.pokemon.com/de-de/gameplay/pokemon-battle-stadium/}{pokemon.com}}
    \label{fig:sword0}
  \end{minipage}%
  \begin{minipage}{.5\textwidth}
    \centering
    \includegraphics[width=.95\linewidth]{images/Sword-1.jpg}
    \captionof{figure}{Fighting another trainer in \\Pokémon Sword \\
      Image source: \href{https://www.nintendo.de/Spiele/Nintendo-Switch/Pokemon-Schwert-1522111.html}{nintendo.de}}
    \label{fig:sword1}
  \end{minipage}
\end{figure}
This Thesis focuses exclusively on the battling aspect of the game as there are detailed lists of locations and
secrets there is to explore within the games. Pokémon battles are turn based where both players, unlike in 
for example chess, make their decisions at the same time. While the core battle mechanics are very simple, 
the game provides a lot of depth which lead to the formation of a strong competitive battling scene.
As catching training Pokémon to a level where they are competitive viable is a time-intensive task, battle
simulators like  the open source project \textit{Pokémon Showdown} \todo{Cite} have arisen. On these fan made
platforms, players have access to all available Pokémon and can compete against each other in a large variety
of formats.
\begin{figure}
  \centering
  \includegraphics[width=1\textwidth]{images/Showdown.png}
  \caption{Battle between two players on \textit{Pokémon Showdown}}
  \label{fig:showdown-battle}
\end{figure}
Figure \ref{fig:showdown-battle} \todo{Move figure to appendix} shows a battle between two players on 
\textit{Pokémon Showdown}. The screenshot was taken from a \textit{Gen 8 random battle} between 
\textit{Buckfae} and another player\footnote{The other name was blurred}. The screenshot was taken
at the start of turn 19. Currently, the \textit{Mewtwo} of \textit{Buckfae} is at full health (indicated
red) while the opposing \textit{Kommo-o} has 33\% of \ac{HP} remaining. Below the avatars of both players,
the remaining team is displayed, marked yellow for player two. While player one has three Pokémon remaining,
his opponent has four members alive. The Pokéball (bottom right) indicates an to the player unknown Pokémon.
Lastly, the blue box highlights the status modifications of the enemy Pokémon. The effects of status conditions
will be covered in detail in section \todo{Link to status}. Below the game window, the possible choices of the
player are displayed. \textit{Mewtwo} has access to the moves \textit{Fire Blast}, \textit{Recover}, \textit{Psystrike}
and \textit{Nastyplot}. In addition to \textit{Dynamaxing} which will be coverd in \todo{Link to dynamax} the player
also has the option to switch to any of his remaining Pokémon \textit{Sirfetch'd} or \textit{Vespiquen}. Lastly,
on the right-hand side a log of the previous turns can be found. In addition to the moves a Pokémon can use,
a Pokémon has one ability and can hold an item that yields advantages in battle. \\
As game states in Pokémon are high-dimensional with the majority of features being both categorical and 
partially observable, Pokémon battles present a worthy challenge for AI to tackle \todo{I stole parts
of this sentence from somewhere, no idea where. Can I keep it?}.
