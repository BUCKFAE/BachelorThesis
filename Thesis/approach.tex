%!TEX root = thesis.tex

\chapter{Approach}
\label{ch:approach}

\section{Basic rules}
\todo{Game is turn based} \\
\todo{Each player has 6 Pokémon} \\
\todo{If a Pokémon has no HP left, it faints} \\
\todo{If all Pokémon of a player fainted, the player loses} 

\subsection{Moves}
\todo{Moves either deal damage or give an advantage later in the battle} 
\subsubsection{Move Categories}
\todo{Physical and Special moves} 

\subsection{Damage calculation}
The damage dealt by a move mainly depends on the \textit{level} of the Pokémon
that uses the move, its effective Attack or Special Attack stat, the
opponent's effective Defense or Special Defense stat and the move's effective
power. 

Precisely, the damage is calculated as follows\cite{Bulbapedia:Damage}:
\begin{dmath}
  \text{Damage} = \left(\frac{\left(\frac{2 \times \text{Level}}{5}\right) \times \text{Power} \times \text{A / D}}{50} + 2 \right)
	\times Targets
	\times Weather
	\times Badge
	\times Critical
	\times random
	\times STAB
	\times Type
	\times Burn
	\times other
\end{dmath}

The only exception for this are moves that deal direct damage. A list 
of these moves can be found at \cite{Bulbapedia:DirectDamage}.

\subsubsection{Level}
\textit{Level} refers to the level of the attacking Pokémon\cite{Bulbapedia:Damage}. 
In Pokémon Showdown, the level is displayed next to the name of the Pokémon.
\todo{Mainline games leveling}

\subsubsection{A / D}
\textit{A} is the effective Attack stat of the attacking Pokémon if the used move is a physical move,
\todo{Reference to physical moves} \\
or the effective Special Attack stat of the attacking Pokémon if the used move is a special move.
\todo{Reference to special moves}
\\
\textit{D} is likewise the effective Defense stat of the target if the used move is a physical move,
or the effective Special Defense of the target if the used move is a special move\cite{Bulbapedia:Damage}.

There are four moves that use stats from different categories, more Information can be found
at \cite{Bulbapedia:MoveStatDifferentCategories}.

\subsubsection{Power}
Power is the effective power of the used move.
\todo{When is the power not equal to the base power}
The \textit{Base Power} of a move in Showdown can be seen when hovering over a move in the move list. \\
\textit{Note:} The same move will always have the same base power. For example, \textit{Fire Punch} will
always have a base power of 75\cite{Bulbapedia:FirePunch}.

\subsubsection{Weather}

\subsection{Effective Stats}



