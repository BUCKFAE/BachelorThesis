%!TEX root = thesis.tex

\chapter{Approach}
\label{ch:approach}

\section{Basic rules}
\todo{Game is turn based} \\
\todo{Each player has 6 Pokémon} \\
\todo{If a Pokémon has no HP left, it faints} \\
\todo{If all Pokémon of a player fainted, the player loses} 

\subsection{Moves}
\todo{Moves either deal damage or give an advantage later in the battle} 
\subsubsection{Move Categories}
\todo{Physical and Special moves} 

\subsection{Damage calculation}
The damage dealt by a move mainly depends on the \textit{level} of the Pokémon
that uses the move, its effective Attack or Special Attack stat, the
opponent's effective Defense or Special Defense stat and the move's effective
power. 

Precisely, the damage is calculated as follows\cite{Bulbapedia:Damage}:
\begin{dmath}
  \text{Damage} = \left(\frac{\left(\frac{2 \times \text{Level}}{5}\right) \times \text{Power} \times \text{A / D}}{50} + 2\right)
	\times Targets
	\times Weather
	\times Badge
	\times Critical
	\times random
	\times STAB
	\times Type
	\times Burn
	\times other
\end{dmath}

The only exception for this are moves that deal direct damage. A list 
of these moves can be found at \cite{Bulbapedia:DirectDamage}.

\subsubsection{Level}
\textit{Level} refers to the level of the attacking Pokémon\cite{Bulbapedia:Damage}. 
In Pokémon Showdown, the level is displayed next to the name of the Pokémon.
\todo{Mainline games leveling}

\subsubsection{A / D}
\textit{A} is the effective Attack stat of the attacking Pokémon if the used move is a physical move,
\todo{Reference to physical moves} \\
or the effective Special Attack stat of the attacking Pokémon if the used move is a special move.
\todo{Reference to special moves}
\\
\textit{D} is likewise the effective Defense stat of the target if the used move is a physical move,
or the effective Special Defense of the target if the used move is a special move\cite{Bulbapedia:Damage}.

There are four moves that use stats from different categories, more Information can be found
at \cite{Bulbapedia:MoveStatDifferentCategories}.

\subsubsection{Power}
Power is the effective power of the used move.
\todo{When is the power not equal to the base power}
The \textit{Base Power} of a move in Showdown can be seen when hovering over a move in the move list. \\
\textit{Note:} The same move will always have the same base power. For example, \textit{Fire Punch} will
always have a base power of 75\cite{Bulbapedia:FirePunch}.

\subsubsection{Weather}

\subsection{Effective Stats}

\section{Hazards}
An \textit{entry hazard} is a condition that affects a side of the field that causes
any Pokémon that is sent into battle on that side of the field to be afflicted by 
a negative effect. Entry hazards are created by moves, usually status moves
\cite{Bulbapedia:EntryHazards}. \\
\todo{This paragraph is copied word by word from Bulbapedia}
\subsection{List of entry hazards}
Currently, there are five moves that create an entry hazard

\subsubsection{Spikes}
\textit{Spikes} is a \textit{Ground}-type entry hazard that causes the opponent
to lose $1/8$\% of their maximum \ac{HP} when they enter the field. This
effect can be stacked up to three times. Two layers of spikes will deal
$1/6$\% and three layers will deal $1/4$\% of the enemies maximum \ac{HP}. \\
\todo{Removal and Immunity of Spikes}
Spikes are created by the move \textit{Spikes}\cite{Bulbapedia:Spikes}.

\subsubsection{Stealth Rock}
\label{sec:stealthrock}
The move \textit{Stealth Rock} sets an entry hazard around the target Pokémon
causing Pokémon on the target's field to receive damage upon being switched in.
The amount of damage inflicted is affected by the effectiveness of the type
\textit{Rock} against the target. Unlike Spikes, this entry hazard does not stack.
The damage taken from the victim's maximum is denoted in table 
\ref{tab:stealth-rock-damage}\cite{Bulbapedia:StealthRock}.
\begin{table}[h]
	\label{tab:stealth-rock-damage}
	\centering
	\begin{tabular}{|c|c|}
		\hline
		\textbf{Type effectiveness} & \textbf{Damage (Max. \ac{HP}}) \\
		\hline 
		0.25x & 3.125\% \\ 
		\hline 
		0.5x &  6.25\% \\ 
		\hline 
		1x & 12.5\% \\
		\hline
		2x & 25\% \\
		\hline
		4x & 50\% \\
		\hline
	\end{tabular} 
	\caption{Damage dealt to Pokémon by Stealth Rocks\cite{Bulbapedia:StealthRock}}
\end{table}
\textit{Note:} Stealth Rocks can also be created by the move \textit{G-Max Stonesurge}.
This damage-dealing Water-type G-Max move is exclusive to Gigantamax Drednaw
\cite{Bulbapedia:GMaxStonesurge}. \\
\todo{Does this move exist in Showdown}

\subsubsection{Sticky Web}
The entry hazard set by the \textit{Bug}-type move \textit{Sticky Web} lowers the
opponents speed stat by one stage upon switching in \cite{Bulbapedia:StickyWeb}. \\
\todo{Pokémon that are not affected by this}

\subsubsection{Poison spikes}
\label{sec:poison-spikes}
\textit{Poison Spikes} set by the \textit{Poison}-type move \textit{Toxic Spikes}
cause the opponent to become poisoned. If two layers of spikes are set, the
Pokémon instead becomes badly poisoned \cite{Bulbapedia:ToxicSpikes}. \\
\todo{Pokémon not affected} \\
\todo{Explain (badly) poisoning}

\subsubsection{Sharp steel}
This entry hazard works very similar to Stealth Rock described in \ref{sec:stealthrock}.
However, Sharp steel can only be set by the \textit{Steel}-type move
\textit{G-Max Steelsurge} which is the exclusive G-Max Move of Gigantamax Copperhead.
The damage dealt by Sharp steel does not stack, the amount of damage dealt is
based on the Type effectiveness of the \textit{Steel}-type against the target.
Exact damage modifiers can be found in table \ref{tab:sharp-steel-damage}
\cite{Bulbapedia:GMaxSteelsurge}.
\begin{table}[h]
	\label{tab:sharp-steel-damage}
	\centering
	\begin{tabular}{|c|c|}
		\hline
		\textbf{Type effectiveness} & \textbf{Damage (Max. \ac{HP}}) \\
		\hline 
		0.25x & 3.125\% \\ 
		\hline 
		0.5x &  6.25\% \\ 
		\hline 
		1x & 12.5\% \\
		\hline
		2x & 25\% \\
		\hline
		4x & 50\% \\
		\hline
	\end{tabular} 
	\caption{Damage dealt to Pokémon by Sharp Steel\cite{Bulbapedia:GMaxSteelsurge}}
\end{table}
\todo{Unaffected Pokémon}

\subsection{Hazard counterplay}
There are some moves that can remove entry hazards. \textit{Rapid Spin} 
\cite{Bulbapedia:RapidSpin} removes entry hazards from the user's side of the field and
\textit{Defog}\cite{Bulbapedia:Defog} removes entry hazards on both sides of the 
field\footnote{In older games \textit{Defog} would only remove Hazards on the
target's side of the field. But as we only investigate the latest version, this
won't be covered in detail.}. In addition, 
\textit{Court Change}\cite{Bulbapedia:CourtChange} will exchange the entry hazards
on each side of the field, along with other one-sided field conditions.
\todo{What other one-sided field conditions are there?}
If a grounded\footnote{The term \textit{grounded} is used to describe a Pokémon that
can't be affected by damaging \textit{Ground}-type moves and several other associated 
effects\cite{Bulbapedia:Grounded}.}
\textit{Poison}-type Pokémon enters the battle, it will remove Toxic 
Spikes, described in \ref{sec:poison-spikes}, from its side of the field.
Lastly, Pokémon holding the item 
\textit{Heavy-Duty Boots}\cite{Bulbapedia:HeavyDutyBoots} are unaffected by
entry hazards, but grounded \textit{Poison}-type Pokémon can still remove
Toxic Spikes even if they hold the boots\cite{Bulbapedia:EntryHazards}.
There are various exceptions and special cases to hazards. 
\todo{Special cases of hazards}