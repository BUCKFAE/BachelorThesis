\documentclass{article}
% General document formatting
\usepackage[margin=0.7in]{geometry}
\usepackage[parfill]{parskip}
\usepackage[utf8]{inputenc}
    
% Related to math
\usepackage{amsmath,amssymb,amsfonts,amsthm}
\title{Proposal \\ Pokemon}
\author{Julian Schubert}

\begin{document}
\maketitle

\section{Introduction}
Pokemon is a video game series created by the Pokemon Company.
The goal of the game is to not only catch pokemon, but also to train them and use
them to battle other trainers. The task of battling optimally is quite interesting
as there is an enourmous amount of possible combinations.

\section{Prerequisites}
Nintendo does not provide an API for the game, however, Pokemon Showdown is a
free online tool that can be used to simulate battles. The tool is available at
https://play.pokemonshowdown.com/. On top of that, the library poke-env is used
on top of that. The library is written in python and provides an easy to use
interface to Showdown. Additionally, the functionality to play games on a 
local server as well as the official Showdown servers is provided.

\section{Battling}
The focus of this thesis will be on batteling. Each player has a team of six 
pokemon. The pokemon are chosen randomly from the available
pokemon in the game. Currently, there are 898 different Pokemon, some of them 
even have different forms. Each pokemon knows 4 different, also randomly assigned,
moves. However, not every Pokemon can learn every move. \\
A move can either damage the opposing pokemon, heal the own pokemon, increase offence
or defence, or add status effects like paralysis or sleep. \\
The game works based on a rock-paper-scissors-like system. Each pokemon has one
or two types, each move has a type as well. Fire-Type moves are strong against
Grass types, Grass types are strong agains Water pokemon and water pokemon
are strong against fire pokemon. In total, there are 18 different types. \\
There are other aspects like weather, speed, base stats and level which
heavily influence the outcome of a battle, they won't be discussed in 
this proposal, however they will have to be taken into account in the thesis.

\section{Execution}
This thesis will investigate multiple possible approaches to optimize battling.
The first approach will be a rule based. Different complexity levels will be
tested against each other. \\
Secondly, backpropagation will be used to train a neural network to play
like a human player. The traning data was provided to me by the pokemon
showdown team, it contains over 8 million replays of random pokemon battles. \\
Lastly, a reinforcement algorithm will be tested. The possibility to pretrain
the network using either rules or replay data will be investigated as well.

\section{Evaluation}
The poke-env libary provides not only a random player, but also a max damage player
that always chooses the move with the highest base damage. A simple reinforcement
approach is given as well. These three agents will be used as baseline. \\
Pokemon Showdown also has an Elo-System similar to chess. The authors of 
Showdown allow bots to compete in ranked matches, so the approaches
developed in this thesis will also be evaluated by playing ranked games 
against actual humans.

\section{Example rulebased approach}
% TODO
\end{document}